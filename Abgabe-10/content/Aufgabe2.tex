\section*{Aufgabe 2: Maximum-Likelihood}
\label{sec:Aufgabe2}

\subsection*{a) Schätzer}
\label{sub:Schätzer}
Ein Schätzer für den Parameter $b$ der Gleichverteilung
\begin{equation}
    f(x)=\begin{cases}
        b^{-1} &0\leq x \leq b\\
        0&\text{sonst}
    \end{cases}
\end{equation}
ist
\begin{equation}
    b=\text{max}(x_i)
    \label{eqn:Schätzer}
\end{equation}

\subsection*{b) Erwartungstreue}
\label{sub:Erwartungstreue}
Für $n\rightarrow \inf$ ist zu erwarten, dass jeder mögliche Wert gewürfelt wird.
Das bedeutet insbesondere, dass $x=b$ gewürfelt wird.
Da $x=b$ der höchstmögliche Wert ist, ist die Gleichung \eqref{eqn:Schätzer} ein erwartungstreuer Schätzer.