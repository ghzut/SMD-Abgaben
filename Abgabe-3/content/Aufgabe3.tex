\section{Aufgabe 3}
\label{sec:Aufgabe3}
Aus der Wahrscheinlichkeitsverteilung
\begin{equation}
  f(x)=
  \begin{cases}
    1   & 0\leq x \leq 1\\
    0   &\text{sonst}
  \end{cases}
\end{equation}
kann man mittels integrieren die Wahrscheinlichkeit ermitteln eine zufällige Zahl
im Intervall $[a,b]\subseteq [0,1)$ (ansonsten ersetzte mit $0$ oder $1$) gilt dann
\begin{equation}
  P(x\in [ a,b ])=\left|\int_a^b f(x)\text{d}x\right|=|b-a|
\end{equation}
\subsection{a)}
Setzen wir nun $a=\frac{1}{3}$ und $b=\frac{1}{2}$ ergibt sich
\begin{equation}
  P\left(x\in \left[\frac{1}{3},\frac{1}{2}\right]\right)=\frac{1}{2}-\frac{1}{3}=\frac{1}{6}
\end{equation}

\subsection{b)}
Wenn wir $a=b$($=\frac{1}{2}$) setzen ergibt sich die Wahrscheinlichkeit als
\begin{equation}
  P\left(x=\frac{1}{2}\right)=P\left(x\in \left[\frac{1}{2},\frac{1}{2}\right]\right)=0
\end{equation}
In Worten gefasst ist es für einen wahren Zufallszahlengenerator nahezu unmöglich irgendeine exakte Zahl tatsächlich zu erzeugen. Zahlen können immer nur in Intervalle fallen.

\subsection{c)}
Da ein Bit nur zwei Werte annehmen kann (und wir eine Mantissenlänge von $23$ annehmen), ergibt sich die trivial
\begin{equation}
  P(\text{Mantisse richtig})=P(M)=2^{-23}=1,19\cdot10^{-7}
\end{equation}
Da $\frac{1}{2}$ durch eine endliche Anzahl von Nachkommastellen dargestellt werden kann, kann sie auch von dem Zufallszahlengenerator mit der eben errechneten Wahrscheinlichkeit geliefert werden.

\subsection{d)}
Die Zahl $\frac{2}{3}$ kann nur approximiert werden und damit auch niemals exakt vom Zufallszahlengenerator erzeugt werden. Ihre Approximation des Computers kann aber erzeugt werden. Dabei ergibt sich die Wahrscheinlichkeit ebenfalls als
\begin{equation}
  P\left(\frac{2}{3}\right)=1,19\cdot10{-7}
\end{equation}
