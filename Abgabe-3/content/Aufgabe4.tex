\section{Aufgabe 4}
\label{sec:Aufgabe4}
Gegeben seien zwei Populationen von jeweils $10 000$ Punkten in einer Ebene. Die Popu-
lation $P_0$ sei eine zweidimensionale, korrelierte Gaussverteilung mit:
\begin{equation}
  \mu_x = 0,\; \mu_y = 3,\; \sigma_x = 3,5,\; \sigma_y = 2,6
\end{equation}
und Korrelation $\rho = 0,9$. Mithilfe der Beziehung
\begin{equation}
  \text{Cor}(x,y)=\rho \sigma_x\sigma_y
\end{equation}
ergibt sich die Kovarianz-Matrix
\begin{equation}
  \text{Cov}_0=\left(\begin{array}{rr}
    \sigma_x^2            & \rho\sigma_x\sigma_y\\
    \rho\sigma_x\sigma_y  & \sigma_y^2
  \end{array}\right)=
  \left(\begin{array}{rr}
    12,25   & 8,19\\
    8,19    & 6,76
  \end{array}\right)
\end{equation}
Die zweite Verteilung $P_1$ ist gegeben durch eine Gaussverteilung in $x$ mit
\begin{equation}
  \mu_x = 6 \; \text{und}\; \sigma_x = 3,5
\end{equation}
und einer Gaussverteilung in $y$, deren Mittelwert linear von $x$ abhängt:
\begin{equation}
  \text{E}[y|x] = \mu_{y|x} = a + bx \; \text{mit} \; a = −0,5, \; b = 0,6\; \text{und} \; \text{Var}[y|x] = \sigma_{y|x}^2 = 1
\end{equation}
Es ergibt mit dem bekannten $\mu_x$
\begin{equation}
  \mu_y=a+b\mu_x=3,1
\end{equation}
Es lässt sich trivial herleiten, dass gilt
\begin{equation}
  \text{Cor}(x,y)=\text{Cor}(x,a+bx)=\text{sign}(b)=1
\end{equation}
Damit ist die Kovarianzmatrix
\begin{equation}
  \text{Cov}_1=\left(\begin{array}{rr}
    \sigma_x^2       & \text{Cor}(x,y)\\
    \text{Cor}(x,y)  & \sigma_y^2
  \end{array}\right)=
  \left(\begin{array}{rr}
    12,25    & 1\\
    1       & 1
  \end{array}\right)
\end{equation}
\subsection{a)}
\begin{figure}
  \includegraphics{build/Aufgabe_4_a.pdf}
  \caption{Verteilungen $P_0$ und $P_1$}
  \label{fig:A4a}
\end{figure}


\subsection{b)}
Folgende Dinge sollen aus den Stichproben berechnet werden:
\begin{align}
  \mu_{\text{Stich}}         &=\left(\left\langle P_x\right\rangle,\left\langle P_y\right\rangle\right)\\
  \sigma_{\text{Stich}}^2    &=\left(\left\langle(P_x-\mu_{\text{Stich},x})^2\right\rangle,\left\langle(P_y-\mu_{\text{Stich},y})^2\right\rangle\right)\\
  \text{Cor}(x,y)            &=\left\langle P_xP_y\right\rangle-\mu_{\text{Stich},x}\mu_{\text{Stich},y}\\
  \text{Cov}_{\text{Stich}}  &=\left(
  \begin{array}{rr}
    \sigma_{\text{Stich},x}^2   & \text{Cor}(x,y)\\
    \text{Cor}(x,y)             & \sigma_{\text{Stich},y}^2
  \end{array}
                            \right)\\
  \rho                       &=\frac{\text{Cor}(x,y)}{\sigma_x\sigma_y}
\end{align}
Für $P_0$ ergibt sich
\begin{align}
  \mu_{\text{Stich}}         &=\left(0,03,\; 3,03\right)\\
  \sigma_{\text{Stich}}      &=\left(3,53, \; 2,60\right)\\
  \text{Cor}(x,y)            &=8,29\\
  \text{Cov}_{\text{Stich}}  &=\left(
  \begin{array}{rr}
    12,46   & 8,29\\
    8,29    & 6,77
  \end{array}
                            \right)\\
  \rho                       &=0,902
\end{align}

Für $P_1$ ergeben sich
\begin{align}
  \mu_{\text{Stich}}         &=\left(5,96,\;  -2,39\right)\\
  \sigma_{\text{Stich}}      &=\left(3,48,\;  1,00\right)\\
  \text{Cor}(x,y)            &=0,993\\
  \text{Cov}_{\text{Stich}}  &=\left(
  \begin{array}{rr}
    12,11   & 0,9993\\
    0,993    & 1,01
  \end{array}
                            \right)\\
  \rho                       &=0,284
\end{align}

Und für die Gesamtverteilung $P_{Ges}$ ergibt sich
\begin{align}
  \mu_{\text{Stich}}         &=\left(2,99,\;  0,319\right)\\
  \sigma_{\text{Stich}}      &=\left(4,59,\;  3,35\right)\\
  \text{Cor}(x,y)            &=-3,39\\
  \text{Cov}_{\text{Stich}}  &=\left(
  \begin{array}{rr}
    21,07   & -3,39\\
    -3,39   & 11,24
  \end{array}
                            \right)\\
  \rho                       &=-0,221
\end{align}
