\section*{Aufgabe 37: Entfaltung: Implementierung}
\label{sec:Aufgabe1}

\subsection*{a)}
\begin{figure}
    \centering
    \includegraphics{build/Attribute.pdf}
    \caption{Alle ausgelesenen Attribute gegen der Energie}
    \label{fig:Attribute}
\end{figure}
Aus Abbildung \ref{fig:Attribute} ist klar erkennbar, dass $Size$ am besten mit der Energie korreliert.

\subsection*{b)}
\begin{figure}
    \centering
    \includegraphics{build/A.pdf}
    \caption{Die Migrationsmatrix $A$}
    \label{fig:A}
\end{figure}


\subsection*{c)}
Mittels des Zusammenhangs
\begin{equation}
    f = A^{-1}\mu_g
\end{equation}
kann die Likelihood-Funktion (und deren Logarithmus) aufgestellt werden.
\begin{align}
    \mathcal{L} &= \prod_{i=1}^n \frac{\mu_{g,i}^{g_i}}{g_i!}e^{-\mu_{g,i}}\nonumber\\
                &= \prod_{i=1}^n \frac{\left(Af\right)_i^{g_i}}{g_i!}e^{-\left(Af\right)_i}
\end{align}

\begin{align}
    -\log\left(\mathcal{L}\right) &= \sum_{i=1}^n -g_i \log\left(Af\right)_i + \left(Af\right)_i + \log\left(g_i!\right)
\end{align}
Mit der Gleichung
\begin{equation}
    \left(Af\right)_i = \sum_{j=1}^m A_{ij}f_j
\end{equation}
kann es vollständig durch die einzelnen Komponenten $f$'s ausgedrückt werden.
\begin{equation}
    -\log\left(\mathcal{L}\right) = \sum_{i=1}^n -g_i \log\left(\sum_{j=1}^m A_{ij}f_j\right) + \sum_{j=1}^m A_{ij}f_j + \log\left(g_i!\right)
\end{equation}

\subsection*{d)}
Fügt man nun den Term $\frac{\tau}{2}||\Gamma f||^2$ (hier wurde $\Gamma=\mathbf{1}_{m\times m}$ gewählt) erhält man
\begin{equation}
    -\log\left(\mathcal{L}\right) = \sum_{i=1}^n -g_i \log\left(\sum_{j=1}^m A_{ij}f_j\right) + \sum_{j=1}^m A_{ij}f_j + \log\left(g_i!\right) + \frac{\tau}{2}||\Gamma f||^2
\end{equation}
Hieraus können nun die folgenden Ableitungen gebildet werden.
\begin{align}
    \left(\left(\nabla\left(-\log\left(\mathcal{L}\right)\right)\right)\left(f\right)\right)_k &= \frac{\partial \left(-\log\left(\mathcal{L}\right)\right)}{\partial f_k}\nonumber\\
                                        &= \sum_{i=1}^n -g_i \left(\sum_{j=1}^m A_{ij}f_j\right)^{-1}\sum_{j=1}^m A_{ij}\delta_{jk} + \sum_{j=1}^m A_{ij}\delta_{jk} + \frac{\tau}{2} \alpha^2 2f_i \delta_{ik}\nonumber\nonumber\\
                                        &= \tau \alpha^2 f_k + \sum_{i=1}^n -g_i \left(\sum_{j=1}^m A_{ij}f_j\right)^{-1} A_{ik} + A_{ik} \nonumber\\
                                        &= \tau \alpha^2 f_k + \sum_{i=1}^n A_{ik} \left(1-\frac{g_i}{\sum_{j=1}^m A_{ij}f_j}\right)
\end{align}

\begin{align}
    \left(H[-\log\left(\mathcal{L}\right)]\left(f\right)\right)_{kl} &= \frac{\partial^2 \left(-\log\left(\mathcal{L}\right)\right)}{\partial f_k \partial f_l}\nonumber\\
                                    &= \frac{\partial \left(\tau \alpha^2 f_k + \sum_{i=1}^n A_{ik} \left(1-\frac{g_i}{\sum_{j=1}^m A_{ij}f_j}\right)\right)}{\partial f_l}\nonumber\\
                                    &= \tau \alpha^2 \delta_{kl} + \sum_{i=1}^n \frac{A_{ik}g_i}{\left(\sum_{j=1}^m A_{ij}f_j\right)^2}\sum_{j=1}^m A_{ij}\delta_{jl}\nonumber\\
                                    &= \tau \alpha^2 \delta_{kl} + \sum_{i=1}^n \frac{A_{ik}g_iA_{il}}{\left(\sum_{j=1}^m A_{ij}f_j\right)^2}
\end{align}

\subsection*{f)}
\begin{figure}
    \centering
    \includegraphics{build/entfaltet.pdf}
    \caption{Die Entfaltung des Testdatensatzes in Abhängigkeit für mehrere Regularisierungsfaktoren $\tau$}
    \label{fig:test}
\end{figure}