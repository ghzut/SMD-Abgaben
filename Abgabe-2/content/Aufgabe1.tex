\section{Aufgabe 1}
\label{sec:Aufgabe1}
In den Zeilen 54-111 der Python-Datei wird die Wahrscheinlichkeit für verschiedene Augenzahlen beim Wurf von zwei Würfeln bestimmt. Dazu wird zunächst bestimmt wie oft die geforderte Augenzahl erreicht wird und dann durch die Anzahl aller Möglichkeiten geteilt.
Für die Aufgabenteile a) - d) werden zwei Würfel mit je 6 Seiten geworfen
also gibt es $N_.1=6^2=36$ mögliche Kombinationen. 
Es ergeben sich die Ausgaben:

\subsection*{a)}

Die Wahrscheinlichkeit, dass die Gesamtaugenzahl genau 9 ist, beträgt $11.11\%$.

\subsection*{b)}

Die Wahrscheinlichkeit, dass die Gesamtaugenzahl 9 oder mehr ist, beträgt $27.78\%$.

\subsection*{c)}

Die Wahrscheinlichkeit, dass der eine Würfel 4 und der andere 5 zeigt, beträgt $5.56\%$.

\subsection*{d)}

Die Wahrscheinlichkeit, dass der rote Würfel 4 und der blaue 5 zeigt, beträgt $2.78\%$.\\

\noindent Für die Teilaufgaben e) - g) liegt das Ergebnis des einen Würfels bereits fest, sodass es nur $N_.2=6$ Möglichkeiten gibt.

\subsection*{e)}

Die Wahrscheinlichkeit, dass, wenn der rote Würfel 4 zeigt, die Gesamtaugenzahl genau 9 ist, beträgt $16.67\%$.

\subsection*{f)}

Die Wahrscheinlichkeit, dass, wenn der rote Würfel 4 zeigt, die Gesamtaugenzahl 9 oder mehr ist, beträgt $33.33\%$.

\subsection*{g)}

Die Wahrscheinlichkeit, dass, wenn der rote Würfel 4 zeigt, der blaue 5 zeigt, beträgt $16.67\%$.