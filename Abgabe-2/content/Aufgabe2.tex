\section{Aufgabe 2}
\label{sec:Aufgabe2}
Die Maxwell'sche Geschwindigkeitsverteilung ist gegeben durch:
\begin{equation}
  f(v)=N\cdot 4\pi v^2 \cdot \text{exp}\left(-\frac{mv^2}{2k_BT}\right)\label{eq:MWG}
\end{equation}
Die Normierungskonstante $N$ kann dadurch bestimmt werden, dass das Integral über alle Wahrscheinlichkeiten gleich 1 sein muss:
\begin{equation}
  \int^{\infty}_.{0} f(v) \text{d}v = \left(\frac{2\pi k_.B T}{m}\right)^{\frac{3}{2}}\cdot N
\end{equation}
Damit ergibt sich
\begin{equation}
  N(m,T)= \left(\frac{m}{2\pi k_.B T}\right)^{\frac{3}{2}}
\end{equation}
und einsetzen in \eqref{eq:MWG}:
\begin{equation}
  f(v)=\left(\frac{m}{2\pi k_.B T}\right)^{\frac{3}{2}}\cdot 4\pi v^2 \cdot \text{exp}\left(-\frac{mv^2}{2k_.BT}\right)
\end{equation}

\subsection*{a)}

Die wahrscheinlichste Geschwindigkeit $v_.m$ erhält man daraus, dass diese Das Maximum der Verteilung beschreibt und somit für die Ableitung gelten muss $f'(v_.m)=0$. 
\begin{align*}
  f'(v) &= \left(\frac{m}{2\pi k_.B T}\right)^{\frac{3}{2}}\cdot 8\pi v \cdot \text{exp}\left(-\frac{mv^2}{2k_.BT}\right)\frac{k_.BT-\frac{mv^2}{2}}{k_.BT}\\
\Rightarrow v_.m &= \sqrt{\frac{2k_.BT}{m}}
\end{align*}


\subsection*{b)}
Der Mittelwert der Geschwindigkeit bestimmt sich über
\begin{equation}
  \langle v \rangle = \int^{\infty}_.{0} v f(v) \text{d}v = \frac{2}{\sqrt{\pi}}\sqrt{\frac{2k_.BT}{m}} = \frac{2}{\sqrt{\pi}} v_.m
\end{equation}

\subsection*{c)}
Zur Bestimmung des Geschwindigkeitsmedians wird zunächst die auf $v_.m$ normierte Geschwindigkeit
\begin{equation}
  \tilde{v} = \frac{v}{v_.m}
\end{equation}
verwendet.
Er bestimmt sich dadurch, dass die Integration über alle $\tilde{v}\leq \tilde{v_.{0,5}}$ genau $\frac{1}{2}$ ergibt.
\begin{align*}
  f\left(\tilde{v}\right) \text{d}\tilde{v} &= \frac{4}{\sqrt{\pi}}\tilde{v}^2 e^{-\tilde{v}^2} \text{d}\tilde{v}\\
  \frac{1}{2} &= \int^{\tilde{v}_.{0,5}}_.{0} \frac{4}{\sqrt{\pi}}\tilde{v}^2 e^{-\tilde{v}^2} \text{d}\tilde{v}
\end{align*}
\noindent Dies kann zu einem Nullstellenproblem umgeformt werden
\begin{equation}
  0=\frac{1}{2} - \int^{\tilde{v}_.{0,5}}_.{0} \frac{4}{\sqrt{\pi}}\tilde{v}^2 e^{-\tilde{v}^2} \text{d}\tilde{v},
\end{equation}
sodass das Newton-Verfahren
\begin{equation}
  x_.{k+1}=x_.k - \frac{f(x_.k)}{f'(x_.k)}
  \label{eqn:Newton}
\end{equation}
angewendet werden kann.\\
Damit ergibt sich mit den Zeilen 117-132 der Python-Datei:
\begin{equation}
  \tilde{v}_.{0,5}=1,08765203176
\end{equation}


\subsection*{d)}
Die Halbwertsbreite ist durch die normierte Geschwindigkeit $\tilde{v}_.{1/2}$ gegeben bei der $f$ auf die Hälfte seines Maximums abgesunken ist:
\begin{equation}
  f(\tilde{v}_.{1/2})=\frac{f(1)}{2}
\end{equation}
Durch das Umformen zu einem Nullstellenproblem
\begin{equation}
  0=2\tilde{v}^2_.{1/2}e^{-\tilde{v}^2_.{1/2}}- e^{-1}
\end{equation}
lässt sich wieder \eqref{eqn:Newton} verwenden.
Die Ableitung ist dabei gegeben durch:
\begin{equation}
  f'(\tilde{v})=-\frac{8 e^{-\tilde{v}^2}\tilde{v}(\tilde{v}^2-1)}{\sqrt{\pi}}
\end{equation}
Es ergeben sich über die Zeilen 135-148 der Python-Datei:
\begin{gather}
  \tilde{v}_.1=0,481623247971\\
  \tilde{v}_.2=1,63656560822\\
  \tilde{v}_.{FWHM}=1,15494236025
\end{gather}

\subsection*{e)}
Die Standardabweichung berechnet sich über:
\begin{equation}
  \sigma^2_.v=\int^{\infty}_.{0} \left(v-\langle v \rangle \right)^2 f(v) \text{d}v
\end{equation}
Somit ergibt sich:
\begin{equation}
  \sigma_.v = \left(\left( \frac{3}{2}- \frac{6}{\pi}\right) \frac{2k_.BT}{m}\right)^{\frac{1}{2}}=\left( \frac{3}{2}- \frac{6}{\pi}\right)^{\frac{1}{2}}\cdot v_.m
\end{equation}


%↓