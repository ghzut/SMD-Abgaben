\section{Aufgabe 2}
\label{sec:Aufgabe2}
Es werden vorerst die Verteilungen
\begin{align}
    P_0&\equiv P\_0\_10000\\
    P_1&\equiv P\_1
\end{align}
verwendet.
\subsection*{a) Mittelwerte}
Die Mittelwerte der Verteilungen ergeben sich zu
\begin{align}
    \mu_0&=\left(\begin{array}{c}-2,13894633\cdot10^{-3}\\2,99221382\end{array}\right)\\
    \mu_1&=\left(\begin{array}{c}6,02103083\\3,09681674\end{array}\right)
\end{align}
\subsection*{b) Kovarianzmatrizen}
Die Kovarianzmatrizen ergeben sich als
\begin{align}
    V_0&=\left(\begin{array}{cc}
        12,36543313 & 8,22931652\\
        8,22931652  & 6,76968479
    \end{array}\right)\\
    V_1&=\left(\begin{array}{cc}
        12,28179162 & 7,37535615\\
        7,37535615  & 5,42946034
    \end{array}\right)
\end{align}
und daraus die kombinierte Matrix
\begin{equation}
    V_{01}=V_0+V_1
\end{equation}
als
\begin{align}
    V_{01}&=\left(\begin{array}{cc}
        24,64722475 & 15,60467267\\
        15,60467267 & 12,19914513
    \end{array}\right)
\end{align}
\subsection*{c) Lineare Fisher-Diskriminante}
Mit der Relation
\begin{equation}
    \vec{\lambda}^*=V_{01}^{-1}\left(\vec{\mu}_0 -\vec{\mu}_1 \right)
    \label{eqn:lambda}
\end{equation}
ergibt sich der Richtungsvektor $\vec{\lambda}^*$. Normiert man diesen ergibt sich der Projektionsvektor $\vec{e}_{\lambda}$
den man als neue Achse $\vec{\lambda}$ verwenden kann mit 
\begin{equation}
    \vec{\lambda}=\lambda\vec{e}_{\lambda}=\lambda\left(\begin{array}{c}
        0,61793837\\
        -0,78622654
    \end{array}\right)
\end{equation}

\subsection*{d) Projektion}
Es ergibt sich die Verteilungen auf die neue Achse projiziert, wie in Abbildung \ref{fig:Projektion_1}.
\begin{figure}
  \includegraphics{build/Aufgabe_2_d.pdf}
  \caption{Die beiden Populationen auf die $\vec{\lambda}$-Achse projiziert}
  \label{fig:Projektion_1}
\end{figure}
\subsection*{e) Reinheit und Effizienz}
Mit den Beziehungen
\begin{align}
    \text{Reinheit}&=\frac{\text{True positive}}{\text{All positives}}    \label{eqn:Reinheit}\\
    \text{Effizienz}&=\frac{\text{True positive}}{\text{All Signals}}     \label{eqn:Effizienz}
\end{align}
ergeben sich die Kurven aus Abbildung \ref{fig:Reinheit_Effizienz}.
\begin{figure}
  \includegraphics{build/Aufgabe_2_e.pdf}
  \caption{Reinheit und Effizienz gegen $\lambda$ geplottet}
  \label{fig:Reinheit_Effizienz}
\end{figure}

\subsection*{f) Signal zu Untergrund}
Wertet man das abgewandelte (um Nullen in der Rechnung zu vermeiden) Verhältnis von Signal
\begin{equation}
    S/B=\frac{\text{Signal} +1}{\text{Untergrund}+1}
    \label{eqn:SB}
\end{equation}
ergibt sich das Maximum bei
\begin{equation}
    \lambda_{S/B_{max}}=-1,8924306178765526
\end{equation}
und der Graph aus Abbildung \ref{fig:SB}
\begin{figure}
  \includegraphics{build/Aufgabe_2_f.pdf}
  \caption{Verhältnis von akzeptiertem Signal zum Untergrund}
  \label{fig:SB}
\end{figure}

\subsection*{g) Signifikanz}
Ähnlich abgewandelt ergibt sich die Signifikanz mittels der Formel
\begin{equation}
    \text{Signifikanz}=\frac{\text{Signal}}{\sqrt{\text{Signal}+\text{Untergrund}+1}}
    \label{eqn:Signifikanz}
\end{equation}
und dessen Maximum bei
\begin{equation}
    \lambda_{Signifikanz_{max}}=-0,5134728115192226
\end{equation}
und der gesamte Graph in Abbildung \ref{fig:Signifikanz}.
\begin{figure}
  \includegraphics{build/Aufgabe_2_g.pdf}
  \caption{Signifikanz}
  \label{fig:Signifikanz}
\end{figure}

\subsection*{h) Schwächeres Signal}
Mit den Werten aus $P\_0\_1000$ als $P_0$ ergeben sich die Werte wie folgt:
\subsubsection*{a) Mittelwerte}
\begin{align}
    \mu_0&=\left(\begin{array}{c}-0,16662573\\2,88552445\end{array}\right)\\
    \end{align}

\subsubsection*{b) Kovarianzmatrizen}
\begin{align}
    V_0&=\left(\begin{array}{cc}
        12,93190847 & 8,67742589\\
        8,67742589  & 7,20625547
    \end{array}\right)\\
    V_{01}&=\left(\begin{array}{cc}
        25,21370009 & 16,05278204\\
        16,05278204 & 12,63571581
    \end{array}\right)
\end{align}

\subsubsection*{c) Lineare Fisher-Diskriminante}
\begin{equation}
    \vec{\lambda}=\lambda\vec{e}_{\lambda}=\lambda\left(\begin{array}{c}
        0,62262353\\
        -0,78252153
    \end{array}\right)
\end{equation}

\subsubsection*{d) Projektion}
Es ergibt sich die Verteilungen auf die neue Achse projiziert, wie in Abbildung \ref{fig:Projektion_1_h}.
\begin{figure}
  \includegraphics{build/Aufgabe_2_hd.pdf}
  \caption{Die beiden Populationen auf die $\vec{\lambda}$-Achse projiziert}
  \label{fig:Projektion_1_h}
\end{figure}

\subsubsection*{e) Reinheit und Effizienz}
Mit den Beziehungen \eqref{eqn:Reinheit} und \eqref{eqn:Effizienz}
ergeben sich die Kurven aus Abbildung \ref{fig:Reinheit_Effizienz}.
\begin{figure}
  \includegraphics{build/Aufgabe_2_he.pdf}
  \caption{Reinheit und Effizienz gegen $\lambda$ geplottet}
  \label{fig:Reinheit_Effizienz_h}
\end{figure}

\subsubsection*{f) Signal zu Untergrund}
Wertet man das abgewandelte (um Nullen in der Rechnung zu vermeiden) Verhältnis von Signal aus Gleichung \eqref{eqn:SB}
ergibt sich das Maximum bei
\begin{equation}
    \lambda_{S/B_{max}}=-1,9063865118976335
\end{equation}
und der Graph aus Abbildung \ref{fig:SB_h}
\begin{figure}
  \includegraphics{build/Aufgabe_2_hf.pdf}
  \caption{Verhältnis von akzeptiertem Signal zum Untergrund}
  \label{fig:SB_h}
\end{figure}

\subsubsection*{g) Signifikanz}
Mit der abgewandelten Signifikanz aus Formel \eqref{eqn:Signifikanz} ergibt sich das Maximum bei
\begin{equation}
    \lambda_{Signifikanz_{max}}=-0,9531090094725752
\end{equation}
und der gesamte Graph in Abbildung \ref{fig:Signifikanz_h}.
\begin{figure}
  \includegraphics{build/Aufgabe_2_hg.pdf}
  \caption{Signifikanz}
  \label{fig:Signifikanz_h}
\end{figure}