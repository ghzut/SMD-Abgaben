\section*{Aufgabe 1: Teilchenidentifikation}
\label{sec:Aufgabe1}
Aus dem Likelihood-Quotienten-Test
\begin{equation}
    \Lambda(X):=\frac{\sup_{\Theta\in\Theta_0}L(X|\Theta)}{\sup_{\Theta\in\Theta_1}L(X|\Theta)}
\end{equation}
lässt sich für unseren Fall eine Wahrscheinlichkeit herleiten (unter der Annahme, dass nur die gegebenen Teilchen auftreten können):
\begin{equation}
    P_{Likelihood}(X)=\frac{L(X)}{\sum_{i\in\Theta}L(i)}
\end{equation}
Mithilfe der $Prior$-Information
\begin{align}
    P_{Prior}(X)=\begin{cases}
        10\%, & X\in[\pi,p] \\
        80\%, & X = K\\
    \end{cases}
\end{align}
lässt sich dann auch eine Abschätzung von Fehlern 1. und 2. Art machen.
\begin{align}
    P_{\text{1. Art}}(X)&=P_{Likelihood}(i\neq X)P_{Prior}(X)\nonumber\\
    P_{\text{2. Art}}(X)&=P_{Likelihood}(X)P_{Prior}(i\neq X)
\end{align}
Diese werden wir im Folgenden jedoch nicht benötigen.

\subsection*{a)}
\begin{equation}
    P_{Likelihood}(X)=\begin{cases}
        6,10\%   & X=\pi\\
        23,47\%  & X=K\\
        70,42\%  & X=p\\
    \end{cases}
\end{equation}

\subsection*{b)}
\begin{equation}
    P_{Likelihood}(X)=\begin{cases}
        78,43\%   & X=\pi\\
        19,61\%   & X=K\\
        1,96\%    & X=p\\
    \end{cases}
\end{equation}

\subsection*{c)}
\begin{equation}
    P_{Likelihood}(X)=\begin{cases}
        3,74\%   & X=\pi\\
        26,74\%  & X=K\\
        69,52\%  & X=p\\
    \end{cases}
\end{equation}