\section*{Aufgabe 3: Ballon-Experiment}
\label{sec:Aufgabe3}
\subsection*{a) Konstanter Fluss}
Die Anzahl der gemessenen Ereignisse folgt einer Poisson-Verteilung mit dem Erwartungswert $\mu$
\begin{equation}
    P(N,T,\mu)=\frac{\mu^N}{N!}e^{-\mu}
\end{equation}
Die Likelihood von $\mu$ gegeben unserer Menge an Messungen $\{N\}$ folgt als
\begin{equation}
    L(\{N\},\mu)=\prod_{i=1}^n \frac{\mu^{N_i}}{N_i!}e^{-\mu}
\end{equation}
Wenn diese nun logarithmiert wird ergibt sich
\begin{equation}
    \mathcal{L}=\sum_{i=1}^n N_i\log(\mu)-\log(N_i!)-\mu
\end{equation}
Um diese zu maximieren muss ihre Ableitung gleich null gesetzt werden und man erhält als Ergebnis
\begin{align}
    0=\frac{\partial \mathcal{L}}{\partial \mu}&=
                \sum_{i=1}^n \left(\frac{N_i}{\mu}-1\right)\nonumber\\
    \mu&=\sum_{i=1}^n \frac{N_i}{n}=\overline{N}\nonumber\\
       &=4218,0
\end{align}

%\begin{align}
%    L&=\prod_{i=1}^n \frac{\overline{N}^{N_i}}{N_i!}e^{-\overline{N}}\nonumber\\
%     &=
%\end{align}

\subsection*{b) Lineares Modell}
Wird nun statt eines konstaten Flusses ein linear ansteigender Fluss angenommen wird der Parameter $\mu$ durch $N_0+AT$ ersetzt ($T$ ist hierbei der Tag an welchem gemessen wurde)
und ergibt sich die neue Wahrscheinlichkeitsverteilung
\begin{equation}
    P(N,T,N_0,A)=\frac{(N_0+AT)^N}{N!}e^{-(N_0+AT)}
\end{equation}
sowie die Likelihood
\begin{equation}
    L(\{N\},\{T\},N_0,A)=\prod_{i=1}^n \frac{(N_0+AT_i)^{N_i}}{N_i!}e^{-(N_0+AT_i)}
\end{equation}
, als  auch die Log-Likelihood
\begin{equation}
    \mathcal{L}=\sum_{i=1}^n N_i\log(N_0+AT_i)-\log(N_i!)-(N_0+AT_i)
\end{equation}
Wird diese numerisch maximiert (siehe python-script) ergeben sich die Parameter
%\begin{align}
%    \frac{\partial\mathcal{L}}{\partial N_0} &= \sum_{i=1}^n \frac{N_i}{N_0+AT_i}-1\\
%    \frac{\partial\mathcal{L}}{\partial A}   &= \sum_{i=1}^n \frac{N_iT_i}{N_0+AT_i}-T_i\\
%\end{align}
\begin{align}
    N_0 &= 4131,33\\
    A   &= 21,67
\end{align}
Beide Parametrisierungen sind in Abbildung \ref{fig:Ballon} dargestellt.
\begin{figure}
    \centering
    \includegraphics{build/Ballon.pdf}
    \caption{Konstantes und lineares Modell an den Messdaten}
    \label{fig:Ballon}
\end{figure}
\subsection*{c) Signifikanz}
Es kann nun der Likelihood-Quotient
\begin{equation}
    \Lambda=\frac{L_{konstant}}{L_{linear}}=\prod_{i=1}^n \left(\frac{\mu}{N_0+AT_i}\right)^{N_i} e^{-\mu+N_0+AT_i}
\end{equation}
gebildet werden und daraus die $\chi^2$-verteilte Grösse
\begin{align}
    -2\log(\Lambda)=3,118
\end{align}
berechnet werden. Der Vergleich mit einer Tabelle ergibt eine Signifikanz von etwa $5\%$.

\subsection*{d) Weiterer Datenpunkt}
Wird nun der zusätzliche Datenpunkt eingefügt ergben sich die Parameter
\begin{equation}
    \mu=4241,0
\end{equation}
und
\begin{align}
    N_0 &= 4140,1\\
    A   &= 29,22
\end{align}
, sowie die Abbildung \ref{fig:Ballon_2}.
\begin{figure}
    \centering
    \includegraphics{build/Ballon_2.pdf}
    \caption{Konstantes und lineares Modell an erweiterten Messdaten}
    \label{fig:Ballon_2}
\end{figure}
Es ergibt sich des weiteren
\begin{equation}
    -2\log(\Lambda)=9,976
\end{equation}
, was einer Signifikanz von knapp $0,1\%$ entspricht.