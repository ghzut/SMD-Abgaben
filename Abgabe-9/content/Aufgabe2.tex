\section*{Aufgabe 2: Teilchenspuren}
\label{sec:Aufgabe2}
Es seien zwei Detektoren entlang der $z$-Achse (an den Positionen $z_1$ und $z_2$ respektiv), die Messungen entlang der $x$-Achse machen ($x_1$ und $x_2$ respektiv).
Hierbei existiert keinen Ablenkung zwischen den zwei Detektoren und die Messunsicherheiten sind bekannt ($\sigma_{x_1}$ und $\sigma_{x_2}$ respektiv) und ohne Korrelation.

\subsection*{Geradengleichung}
\label{sub:Geradengleichung}
Die Geradengleichung
\begin{equation}
    x=az+b\label{eqn:Geradengleichung}
\end{equation}
wird trivialerweise durch
\begin{align}
    a&=\frac{x_1-x_2}{z_1-z_2}\nonumber\\
    b&=\frac{x_2 z_1 - x_1 z_2}{z_1-z_2}
\end{align}
eindeutig gelöst.
Dargestellt über die Transformationsmatrix ist die Lösung
\begin{equation}
    \left(
        \begin{array}{r}
            a\\
            b
        \end{array}
    \right)
    = B \left(
            \begin{array}{r}
                x_1\\
                x_2
            \end{array}
        \right) =
    \left(
        \begin{array}{rr}
            (z_1-z_2)^{-1}          & -(z_1-z_2)^{-1}\\
            \frac{-z_2}{z_1-z_2}    & \frac{z_1}{z_1-z_2}
        \end{array}
    \right)
    \left(
        \begin{array}{r}
        x_1\\
        x_2
        \end{array}
    \right)
\end{equation}
mit der Kovarianzmatrix
\begin{equation}
    V(x)=
    \left(
        \begin{array}{rr}
            \sigma_{x_1}^2 & 0\\
            0 & \sigma_{x_2}^2
        \end{array}
    \right)
\end{equation}
Hieraus ergibt sich mit der Beziehung
\begin{equation}
    V(a,b)=BV(x)B^T
\end{equation}
die Kovarianzmatrix für die Parameter $a$ und $b$
\begin{equation}
    V(a,b)=
    \left(
        \begin{array}{rr}
            \frac{\sigma_{x_1}^2+\sigma_{x_2}^2}{(z_1-z_2)^2} & -\frac{\sigma_{x_1}^2z_2+\sigma_{x_2}^2z_1}{(z_1-z_2)^2} \\
            -\frac{\sigma_{x_1}^2z_2+\sigma_{x_2}^2z_1}{(z_1-z_2)^2} & \frac{\sigma_{x_1}^2z_2^2+\sigma_{x_2}^2z_1^2}{(z_1-z_2)^2}
        \end{array}
    \right)
    = 
    \left(
        \begin{array}{rr}
            \sigma_a^2 & \text{cov}(a,b)\\
            \text{cov}(a,b) & \sigma_b^2
        \end{array}
    \right)
\end{equation}
und der Korrelationsparamter
\begin{equation}
    \rho = -\frac{\sigma_{x_1}^2z_2+\sigma_{x_2}^2z_1}{\sqrt{z_2^2\left(\sigma_{x_1}^4+\sigma_{x_1}^2\sigma_{x_2}^2 \right) + z_1^2\left(\sigma_{x_2}^4+\sigma_{x_1}^2\sigma_{x_2}^2 \right)}}
\end{equation}

\subsection*{Vorhersage}
\label{sub:Vorhersage}
Den Erwartungswert für eine Messung an einem dritten Detektor ($z_3$ und $x_3$ respektiv) setzten wir die Werte in die Gleichung \eqref{eqn:Geradengleichung} ein
\begin{equation}
    x_3=\frac{(x_1-x_2)z_3+x_2 z_1 - x_1 z_2}{z_1-z_2}
\end{equation}
Für die Berechnung des Fehlers benutzten wir wieder die Gleichung \eqref{eqn:mit_korr} und erhalten
\begin{equation}
    \sigma_{x_3}=\sqrt{z_3^2\sigma_a^2+\sigma_b^2+2 z_3 \text{cov}(a,b)}
\end{equation}
\subsection*{Weglassen der Korrelation}
\label{sub:Weglassen der Korrelation}
Wenn die Korrelation von $a$ und $b$ nicht berücksichtigt wird, ergibt sich stattdessen
\begin{equation}
    \sigma_{x_3}=\sqrt{z_3^2\sigma_a^2+\sigma_b^2}
\end{equation}
Da cov$(a,b) < 0$ ist, produziert diese Fehlerformel größere Fehler als die korrekte.