\section*{Aufgabe 1: Fehlerfortpflanzung}
\label{sec:Aufgabe1}
Betrachtet wird die Ausgleichsgerade
\begin{equation}
    y=a_0+a_1x
\end{equation}
mit den Parametern
\begin{align}
    a_0&=1,0\pm 0,2\nonumber\\
    a_1&=1,0\pm 0,2\nonumber\\
    \rho = -0,8
\end{align}
\subsection*{Analytisch}
\label{sub:Analytisch}
Wird nun die Fehlerfortpflanzung auf $y$ ohne Korrelation errechnet ergibt sich Gleichung \eqref{eqn:ohne_korr}
\begin{align}
    \sigma_y&=\sqrt{\sum_{i=1}^n\left(\frac{\partial y}{\partial x_i}\sigma_{x_i}\right)^2}\nonumber\\
            &=\sqrt{\left(1\cdot\sigma_{a_0}\right)^2+\left(x\sigma_{a_1}\right)^2}\nonumber\\
            &=0,2\sqrt{1+x^2}
            \label{eqn:ohne_korr}
\end{align}
Mit Korrelation ergibt sich dann Gleichung \eqref{eqn:mit_korr}
\begin{align}
    \sigma_y&=\sqrt{\sum_{i=1}^n\left(\frac{\partial y}{\partial x_i}\sigma_{x_i}\right)^2+2\sum_{i=1}^{n-1}\sum_{k=i+1}^{n}\left(\frac{\partial y}{\partial x_i}\right)\left(\frac{\partial y}{\partial x_k}\right)\cdot cov(x_i,x_k)}\nonumber\\
            &=\sqrt{0,04+0,04x^2+2 x (-0,8)\cdot 0,04}\nonumber\\
            &=0,2\sqrt{1+x^2-1,6x}
            \label{eqn:mit_korr}
\end{align}

\subsection*{Numerisch}
\label{sub:Numerisch}
Die Parameter können auch numerisch simuliert werden. Mit entsprechendem seed ergibt sich dann der Scatterplot \ref{fig:A_1_b_1}
\begin{figure}
    \centering
    \includegraphics{build/Aufgabe_21_b_1.pdf}
    \caption{Scatterplot der Parameter $a_0$ und $a_1$}
    \label{fig:A_1_b_1}
\end{figure}
Mittels dieser simulierten Parametern kann dann die Unsicherheit von $y$ in Abhängigkeit von $x$, wie in Abbildung \ref{fig:A_1_b_2}, berechnen lassen.
\begin{figure}
    \centering
    \includegraphics{build/Aufgabe_21_b_2.pdf}
    \caption{Plot der Unsicherheit von $y$ in Abhängigkeit von $x$}
    \label{fig:A_1_b_2}
\end{figure}
Anhand dieses Plot ist bereits erkennbar, dass die numerisch berechnete Unsicherheit für Werte weiter weg von $x=0$ sich immer stärker von der exakt berechneten unterscheidet.

\subsection*{Auswertung an speziellen X}
\label{sub:Auswertung an speziellen X}
Werden aber Funktionswert mit Unsicherheit zusammen ausgewertet, ergibt sich, dass, auch wenn nicht identisch mit den exakten Werten,
die numerischen Werte mit ihren Fehlerintervallen sich mit den Fehlerintervalle der exakten Werte überlappt.
\begin{align}
    x  &=  -3 :\nonumber\\
        y_{\text{Numerisch}}    &=  -1,67 +/- 0,63\nonumber\\
        y_{\text{Exakt}}        &=  -2 +/- 0,77\nonumber\\
    x  &=  0 :\nonumber\\
        y_{\text{Numerisch}}    &=  1,07 +/- 0,14\nonumber\\
        y_{\text{Exakt}}        &=  1 +/- 0,2\nonumber\\
    x  &=  3 :\nonumber\\
        y_{\text{Numerisch}}    &=  3,82 +/- 0,35\nonumber\\
        y_{\text{Exakt}}        &=  4 +/- 0,47\nonumber\\
\end{align}