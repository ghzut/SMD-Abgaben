\section*{Aufgabe 3: Stichprobenvarianz}
\label{sec:Aufgabe3}

\subsection*{Arithmetisches Mittel}
\label{sub:Arithmetisches Mittel}
Wird der Erwartungswert von
\begin{equation}
    \bar{x}=\frac{\sum_{i=1}^nx_i}{n}
\end{equation}
gebildet, ergibt sich
\begin{equation}
    \text{E}(\bar{x})=\text{E}\left(\frac{\sum_{i=1}^nx_i}{n}\right)=\frac{\sum_{i=1}^n\text{E}(x_i)}{n}=\frac{n\mu}{n}=\mu
\end{equation}
Das arithmetische Mittel ist somit erwartungstreu.

\subsection*{Varianz}
\label{sub:Varianz}
Die Varianz von arithmetischen Mittel ist definiert als
\begin{align}
    \text{E}\left((\bar{x}-\mu)^2 \right)&=\text{Var}(\bar{x})=\text{Var}\left(\bar{x}=\frac{\sum_{i=1}^nx_i}{n}\right)\\
    &=\bar{x}=\frac{\text{Var}\left(\sum_{i=1}^nx_i\right)}{n^2}=\bar{x}=\frac{\sum_{i=1}^n\text{Var}(x_i)}{n^2}\\
    &=\frac{n\sigma^2}{n^2}=\frac{\sigma^2}{n}
\end{align}
Sie ist somit erwartungstreu.

\subsection*{Schätzer 0-ter Art}
\label{sub:Schätzer 0-ter Art}
Von der Schätzfunktion für $\sigma^2$
\begin{equation}
    S_0^2=\frac{\sum_{i=1}^n(x_i-\mu)^2}{n}
\end{equation}
wird der Erwartungswert gebildet:
\begin{align}
    \text{E}(S_0^2)&=\frac{\text{E}\left(\sum_{i=1}^nx_i^2-2x_i\mu +\mu^2\right)}{n}\\
    &=\frac{\sum_{i=1}^n\text{E}(x_i^2)-2\text{E}(x_i)\mu +\mu^2}{n}\\
    &=\frac{n(\sigma^2+\mu^2)-2n\mu^2 +n\mu^2}{n}\\
    &=\sigma^2
\end{align}
$S_0$ ist somit erwartungstreu.