\section{Aufgabe 1}
\label{sec:Aufgabe1}

\subsection*{a) Signal}
Der Fluss der Neutrinos ist gegeben durch
\begin{equation}
    \Phi = \Phi_0 \left(\frac{E}{\text{TeV}}\right)^{-\gamma}
\end{equation}
Hierbei ist $\frac{E}{Tev}:=x$ die zufallsverteilte Variable und $\frac{\Phi}{\Phi_0}=Nx^{-\gamma}$ die unnormierte Wahrscheinlichkeitsdichte auf dem Intervall $E\in [1,\infty)$.
Die Verteilungsfunktion ist somit gegeben als
\begin{equation}
    F(x)=N\frac{x^{1-\gamma}}{1-\gamma}+C
\end{equation}
Die Bestimmung der Normierungskonstante $N$ ergibt sich aus
\begin{align}
    1=F(\infty)-F(1)&=N\left(0-(1-\gamma)^{-1}\right)\\
    \Rightarrow N&=\gamma-1
\end{align}
und damit
\begin{equation}
    F(x)=-x^{1-\gamma} + C
\end{equation}
Die Bestimmung der Randwertkonstante $C$ erhält man aus
\begin{align}
    0&=F(1)=1+C\\
    \Rightarrow C&=-1
\end{align}
und damit
\begin{equation}
    F(x)=1-x^{1-\gamma}
\end{equation}
Daraus ergibt sich die Berechnung der Zufallszahl $x$ mit der gleichverteilten Zufallszahl $Y\in [0,1]$
\begin{equation}
    x=(1-Y)^{\frac{1}{1-\gamma}}
\end{equation}
Es werden im Folgenden $10^5$ Ereignisse simuliert.
\subsection*{b) Akzeptanz}
Für eine gegebene Energie ist die Wahrscheinlichkeit das Neutrino zu detektieren gegeben als
\begin{equation}
    P(E)=\left(1-e^{-\frac{E}{2}}\right)^3
    \label{eqn:Akzeptanz}
\end{equation}
Rollt man für jedes Neutrino nun einen Zufallszahlengenerator und vergleich, ob die Zahl über oder unter dem Graphen von Funktion \eqref{eqn:Akzeptanz} liegt, erhält man eine Verteilung von akzeptierten und nicht akzeptierten Ereignissen.
Diese ist in Abbildung \ref{fig:Akzeptanz} dargestellt.
\begin{figure}
    \centering
    \includegraphics{build/Akzeptanz.pdf}
    \caption{Zufällige Akzeptanzzahl für jedes Ereignis aufgetragen gegenüber der Energie}
    \label{fig:Akzeptanz}
\end{figure}

\subsection*{c) Detektierte Hits}
Zur Energiemessung wird eine mit der Energie korrelierte Variable ausgewertet (hier: Hits). Die Anzahl der Hits werden als Normalverteilung mit $\mu=10\frac{E}{\text{Tev}}$ und $\sigma=2\frac{E}{\text{Tev}}$ angenommen.
Es wird trivialerweise vorausgesetzt, dass nur ganzzahlige (runden) Werte grösser 1 akzeptiert werden.
Zur Realisierung der Normalverteilung wird hierzu die Polarmethode verwendet:

Erzeuge zwei gleichverteilte Zufallszahlen $u_1,u_2\in[0,1]$ und setze
\begin{align}
    v_1&=2u_1-1\\
    v_2&=2u_2-1
\end{align}
setze
\begin{equation}
    s=v_1^2+v_2^2
\end{equation}
und verwerfe falls
\begin{equation}
    s\geq 1
\end{equation}
Die Zahlen
\begin{align}
    x_1&=v_1\sqrt{-\frac{2}{s}\text{ln}s}\\
    x_2&=v_2\sqrt{-\frac{2}{s}\text{ln}s}
\end{align}
sind nun zwei unabhängige, normalverteilte Zufallszahlen mit $\mu=1$ und $\sigma=1$.
Transformiert man diese nach
\begin{equation}
    \tilde{x}=x\sigma + \mu
\end{equation}
erhält man Zufallszahlen mit den gewünschten Mittelwerten und Standardabweichungen.
Da nur Ergebnisse grösser 1 akzeptiert werden sollen, müssen eventuell neue Zahlen gezogen werden.
\subsection*{d) Ortsmessung}
Ein Detektor besitzt nur eine endliche Ortsauflösung, die von der Anzahl der Hits in einem Ereignis abhängt.
Es wird hier angenommen quadratisch mit einer Kantenlänge von 10 Längeneinheiten ist und das Signal am Punkt $(7,3)$ auf den Detektor trifft.
Die Ortsmessung wird als Normalverteilung in beiden Koordinaten mit einer Standardabweichung des detektierten Ortes
\begin{equation}
    \sigma=\left(\text{log}_{10}(N+1)\right)^{-1}
\end{equation}
angenommen. Die Simulation ergibt dann die Verteilung in Abbildung \ref{fig:Orte_Signal}.
\begin{figure}
    \centering
    \includegraphics{build/Orte_Signal.pdf}
    \caption{Die detektierten Orte des simulierten Signals in ein 2D-Histogramm}
    \label{fig:Orte_Signal}
\end{figure}

\subsection*{e) Untergrund}
Die Untergrundereignisse werden als sehr viel zahlreicher angenommen. Es werden im Folgenden $10^7$ Untergrundereignisse simuliert.
Der Zehner-Logarithmus der Anzahl der Hits pro Ereignis wird als Normalverteilung mit $\mu=2$ und $\sigma=1$ angenommen.
Die x- und y-Koordinate sind um den Mittelpunkt $(5,5)$ des Detektors mit $\sigma=3$ normalverteilt. Es besteht eine Korrelation $\rho=0,5$ zwischen den Achsen.
Korrelierte normalverteilte Variablen ergeben sich als
\begin{align}
    x&=\sqrt{1-\rho^2}\sigma x'+\rho\sigma y'+\mu\\
    y&=\sigma y'+\mu
\end{align}
mit $x',y'$ als unkorrelierten normalverteilten Zufallszahlen.
Die Simulation des Untergrunds ergeben die Ortsintensität in Abbildung \ref{fig:Orte_Untergrund} und der Zehnerlogarithmus der Anzahl der Hits in Abbildung \ref{fig:Hits_Untergrund}.
Das Ausgefranste am linken Ende der Abbildung \ref{fig:Hits_Untergrund} liegt an der Ganzzahligkeit der Hits.
\begin{figure}
    \centering
    \includegraphics{build/Orte_Untergrund.pdf}
    \caption{Die detektierten Orte des simulierten Untergrunds in einem 2D-Histogramm}
    \label{fig:Orte_Untergrund}
\end{figure}
\begin{figure}
    \centering
    \includegraphics{build/Hits_Untergrund.pdf}
    \caption{Zehnerlogarithmus der Anzahl der Hits des simulierten Untergrunds in einem Histogramm}
    \label{fig:Hits_Untergrund}
\end{figure}