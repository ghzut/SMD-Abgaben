% Unterstützung für Links und PDF Metadaten
\input{header.tex}
% Einstellungen hier, z.B. Fonts

\begin{document}

\section{Numerische Stabiltät}

\begin{gather}
  f(x)=\left(x^3+\frac{1}{3}\right)-\left(x^3-\frac{1}{3}\right)\label{eqn:f}\\
  g(x)=\frac{\left(3+\frac{x^3}{3}\right)-\left(3-\frac{x^3}{3}\right)}{x^3}\label{eqn:g}
\end{gather}

Für Formel \eqref{eqn:f} ist der Fehler kleiner als $1\%$ für x-Werte im Bereich von $-41285 \sim 41285$.
Für Zahlen $|x|>165141$ ist die Gleichung numerisch $=0$.

\begin{figure}
  \centering
  \includegraphics[width=\textwidth]{build/FehlerFlog.pdf}
  \caption{Relativer Fehler von $f(x)$ im positiven Bereich (logarithmisch)}
  \label{fig:FehlerFlog}
\end{figure}

Für Werte $x>4.01269392415 \cdot 10^{-5}$ ist der numerische Fehler der Formel \eqref{eqn:g} kleiner $1\%$.
Für Zahlen $x<8.73291858992 \cdot 10^{-6}$ ist die Gleichung numerisch $=0$.

\begin{figure}
  \centering
  \includegraphics[width=\textwidth]{build/FehlerGlog.pdf}
  \caption{Relativer Fehler von $g(x)$ im positiven Bereich (logarithmisch)}
  \label{fig:FehlerGlog}
\end{figure}

\section{\texorpdfstring{$\symup{e}^-\symup{e}^+\rightarrow\gamma \gamma$}{Paarvernichtung}}

\begin{gather}
    \frac{\symup{d}\sigma}{\symup{d}\Omega} = \frac{\alpha^2}{s} \left(\frac{2+\text{sin}(\Theta)}{1-\beta^2 \text{cos}(\Theta)}\right)^2\label{eqn:WQStandard}\\
    \frac{\symup{d}\sigma}{\symup{d}\Omega} =  \frac{\alpha^2}{s} \frac{2+\text{sin}^2(\Theta)}{\frac{1}{\gamma^2}+\beta^2 \text{sin}^2(\Theta)}\label{eqn:WQModifiziert}
\end{gather}


\begin{figure}
  \centering
  \includegraphics[width=\textwidth]{build/plot1.pdf}
  \caption{\eqref{eqn:WQStandard} und \eqref{eqn:WQModifiziert} um $\theta = 0$}
  \label{fig:t0}
\end{figure}

\begin{figure}
  \centering
  \includegraphics[width=\textwidth]{build/plot2.pdf}
  \caption{\eqref{eqn:WQStandard} und \eqref{eqn:WQModifiziert} um $\theta = \frac{\pi}{2}$}
  \label{fig:tpi2}
\end{figure}

\begin{figure}
  \centering
  \includegraphics[width=\textwidth]{build/plots.pdf}
  \caption{\eqref{eqn:WQStandard} und \eqref{eqn:WQModifiziert} um $\theta = \pi$}
  \label{fig:tpi}
\end{figure}

\end{document}
