% Unterstützung für Links und PDF Metadaten
\input{header.tex}
% Einstellungen hier, z.B. Fonts

\begin{document}

\section{Numerische Stabiltät}
Die beiden Funktionen
\begin{gather}
  f(x)=\left(x^3+\frac{1}{3}\right)-\left(x^3-\frac{1}{3}\right)\label{eqn:f}\\
  g(x)=\frac{\left(3+\frac{x^3}{3}\right)-\left(3-\frac{x^3}{3}\right)}{x^3}\label{eqn:g}
\end{gather}
sind beide $=2/3$.

\subsection{Untersuchung von Gleichung \eqref{eqn:f}}
\noindent Für Formel \eqref{eqn:f} lässt sich mittels der Zeilen $21-32$ in der Python-Datei bestimmen, dass der Fehler kleiner als $1\%$ ist für $-41285 \leq x \leq 41285$. Die Darstellung des relativen Fehlers ist dabei in Abbildung \ref{fig:FehlerF}, eine logarithmische Darstellung des positiven Bereichs in Abbildung \ref{fig:FehlerFlog} zu sehen.\\
Für Zahlen $|x|>165141$ ist die Gleichung numerisch $=0$.

\begin{figure}
  \centering
  \includegraphics[width=\textwidth]{build/A1_1.pdf}
  \caption{Relativer Fehler von $f(x)$ im Bereich $-41285 \leq x \leq 41285$}
  \label{fig:FehlerF}
\end{figure}

\begin{figure}
  \centering
  \includegraphics[width=\textwidth]{build/A1_12.pdf}
  \caption{Relativer Fehler von $f(x)$ im Bereich $-41285 \leq x \leq 41285$}
  \label{fig:FehlerFlog}
\end{figure}

\newpage

\subsection{Untersuchung von Gleichung \eqref{eqn:g}}
\noindent Mittels der Zeilen $53-58$ ist ermittelbar, daas für Werte $x>4.01269392415 \cdot 10^{-5}$ der numerische Fehler der Formel \eqref{eqn:g} kleiner als $1\%$ ist.
Für Zahlen $x<8.73291858992 \cdot 10^{-6}$ ist die Gleichung numerisch $=0$.

\begin{figure}
  \centering
  \includegraphics[width=\textwidth]{build/A1_2.pdf}
  \caption{Relativer Fehler von $g(x)$ im positiven Bereich (logarithmisch)}
  \label{fig:FehlerGlog}
\end{figure}

\newpage

\section{\texorpdfstring{$\symup{e}^-\symup{e}^+\rightarrow\gamma \gamma$}{Paarvernichtung}}
\subsection{Wirkungsquerschnitt}
Für große Werte von $\gamma=\frac{E_.e}{m_.ec^2}$ und damit $\beta\approx 1$ ist die Formel für den Wirkungsquerschnitt \eqref{eqn:WQStandard} im Bereich um $\Theta=0$ instabil, da im Nenner die Gefahr einer Auslöschung besteht. Generell ist der Nenner, da er für $\Theta\leq \frac{\pi}{2}$ immer $<1$ ist, eine Gefahr für Instabilität, weil so durch eine kleine Zahl geteilt wird.\\
Beheben lässt sich dies durch Umformungen mit den Beziehungen

\begin{gather}
  1-\beta^2=\frac{1}{\gamma^2}\nonumber\\
  1=\sin^2(\Theta)+\cos^2(\Theta)\nonumber
\end{gather}
zu Formel \eqref{eqn:WQModifiziert}.

\begin{align}
    \frac{\mathrm{d}\sigma}{\mathrm{d}\Omega} &= \frac{\alpha^2}{s} \left(\frac{2+\sin(\Theta)}{1-\beta^2 \cos(\Theta)}\right)^2\label{eqn:WQStandard}\\
&= \frac{\alpha^2}{s}\frac{2+\sin^2(\Theta)}{\frac{1}{\gamma^2}\cos^2(\Theta)+\sin^2(\Theta)}\label{eqn:WQModifiziert}
\end{align}
Hierbei ist

\begin{gather}
  s=(2E_\text{e})^2 \;\;(E_\text{e} \text{ ist die Energie der Teilchen im Schwerpunktssystem})\text{.}\nonumber
\end{gather}

\noindent In den Abbildungen \ref{fig:t0} bis \ref{fig:tpi}, die im Code über die Zeilen $74$-$122$ erzeugt werden, sind die Bereiche um $\Theta=0$, $\Theta=\frac{\pi}{2}$ und $\Theta=\pi$ zu sehen.
Daran lassen sich kaum Unterschiede ablesen, auch weil der Nenner nach wie vor sehr klein ist. Trägt man allerdings in Abbildung \ref{fig:Abweichung} die Differenz zwischen den beiden Versionen gegen $\Theta$ auf, lässt sich erkennen, das diese sich für Werte um $\Theta=0$ bzw. $2\pi$ deutlich unterscheiden.

\begin{figure}
  \centering
  \includegraphics[width=\textwidth]{build/plot1.pdf}
  \caption{\eqref{eqn:WQStandard} und \eqref{eqn:WQModifiziert} um $\Theta = 0$}
  \label{fig:t0}
\end{figure}

\begin{figure}
  \centering
  \includegraphics[width=\textwidth]{build/plot2.pdf}
  \caption{\eqref{eqn:WQStandard} und \eqref{eqn:WQModifiziert} um $\Theta = \frac{\pi}{2}$}
  \label{fig:tpi2}
\end{figure}

\begin{figure}
  \centering
  \includegraphics[width=\textwidth]{build/plot3.pdf}
  \caption{\eqref{eqn:WQStandard} und \eqref{eqn:WQModifiziert} um $\Theta = \pi$}
  \label{fig:tpi}
\end{figure}

\begin{figure}
  \centering
  \includegraphics[width=\textwidth]{build/Differenz.pdf}
  \caption{\eqref{Die Differenz von eqn:WQStandard} und \eqref{eqn:WQModifiziert} im Bereich $0 \leq \Theta \leq 2\pi$}
  \label{fig:Abweichung}
\end{figure}

\newpage

\subsection{Konditionszahl}
Aus der Beziehung für die Konditionszahl
\begin{equation}
  K(x):=\left|x\frac{f'(x)}{f(x)}\right|
\end{equation}
ergibt sich die Formel \eqref{eqn:K} und damit den Graphen \ref{fig:kondition}.
\begin{equation}
  K(\Theta)= \Theta  \left|\frac{2\sin(\Theta)\cos(\Theta)\left(1-3\beta^2\right)}{\left(2+\sin^2(\Theta)\right)\left(1-\beta^2\cos^2(\Theta)\right)}\right| \label{eqn:K}
\end{equation}

\begin{figure}
  \centering
  \includegraphics{build/kondition.pdf}
  \caption{Konditionszahl in Abhängigkeit des Winkels}
  \label{fig:kondition}
\end{figure}
\noindent Aus dem Plot \ref{fig:kondition}, der in den Zeilen $140$-$146$ erzeugt wird, ist erkennbar, dass das Problem im Bereich um $180\si{\degree}$ schlecht konditioniert ist.
Überall sonst ist es sehr gut konditioniert.

\end{document}



%☺¶
