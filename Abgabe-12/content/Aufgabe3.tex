\section*{Aufgabe 3: Entfaltung in zwei Intervallen}
\label{sec:Aufgabe3}
\subsection*{a)}
\begin{equation}
    \mathbf{A} =\left(\begin{matrix}
                    1 - \epsilon & \epsilon\\
                    \epsilon     & 1 - \epsilon
                 \end{matrix}\right)
\end{equation}

\begin{equation}
    \mathbf{g} = \mathbf{A f}
\end{equation}

\subsection*{b)}
\begin{equation}
    \mathbf{A}^{-1} = \frac{1}{2\epsilon-1} \left(\begin{matrix}
                    \epsilon - 1 & \epsilon\\
                    \epsilon     & \epsilon - 1
                 \end{matrix}\right)
\end{equation}
\begin{equation}
    \mathbf{f} = \frac{1}{2\epsilon-1} \left(\begin{matrix}
                    \epsilon - 1 & \epsilon\\
                    \epsilon     & \epsilon - 1
                 \end{matrix}\right) \mathbf{g}
\end{equation}

\subsection*{c)}
\begin{align}
    \mathbf{V}[\mathbf{f}]
        &=  \mathbf{A^{-1}V}[\mathbf{g}]\mathbf{\left(A^{-1}\right)^T}\nonumber\\
        &=  \frac{1}{2\epsilon-1} \left(\begin{matrix}
                    \epsilon - 1 & \epsilon\\
                    \epsilon     & \epsilon - 1
                 \end{matrix}\right)
            \left(\begin{matrix}
                \sigma_{g_1}^2 & 0\\
                0 & \sigma_{g_2}^2
            \end{matrix} \right)
            \frac{1}{2\epsilon-1} \left(\begin{matrix}
                    \epsilon - 1 & \epsilon\\
                    \epsilon     & \epsilon - 1
                 \end{matrix}\right)\nonumber\\
        &=  \frac{1}{(2\epsilon-1)^2} \left(\begin{matrix}
                \epsilon -1 & \epsilon\\
                \epsilon & \epsilon -1
            \end{matrix} \right)
            \left(\begin{matrix}
                (\epsilon-1)\sigma_{g_1}^2 & \epsilon \sigma_{g_1}^2\\
                \epsilon \sigma_{g_2}^2 & (\epsilon-1)\sigma_{g_2}^2
            \end{matrix} \right)\nonumber\\
        &=  \frac{1}{(2\epsilon-1)^2} \left(\begin{matrix}
                (\epsilon-1)^2\sigma_{g_1}^2 + \epsilon^2 \sigma_{g_2}^2 & \epsilon (\epsilon-1) (\sigma_{g_2}^2+\sigma_{g_1}^2)\\
                \epsilon (\epsilon-1) (\sigma_{g_2}^2+\sigma_{g_1}^2) & (\epsilon-1)^2\sigma_{g_2}^2 + \epsilon^2 \sigma_{g_1}^2
            \end{matrix} \right)
\end{align}

\subsection*{d)}
\begin{equation}
    f\left(\left(\begin{matrix}
            200\\
            169
        \end{matrix}\right),\; 0,1 \right) = \left(\begin{matrix}
                                                    203,875\\
                                                    165,125
                                                \end{matrix}\right)
\end{equation}

\begin{equation}
    V = \left(\begin{matrix}
            255,77 & -51,89\\
            -51,89 & 217,02
        \end{matrix}\right)
\end{equation}

\begin{equation}
    \sigma_f = \left(\begin{matrix}
                    15,99\\
                    14,73
                \end{matrix}\right)
\end{equation}

\begin{equation}
    \rho = -0,220
\end{equation}

\subsection*{e)}

\begin{equation}
    f(\left(\begin{matrix}
            200\\
            169
        \end{matrix}\right) ,\; 0,4 ) = \left(\begin{matrix}
                                            262\\
                                            107
                                        \end{matrix}\right)
\end{equation}


\begin{equation}
    V = \left(\begin{matrix}
            2476  & -2214\\
            -2214 & 2321
        \end{matrix}\right)
\end{equation}

\begin{equation}
    \sigma_f = \left(\begin{matrix}
                    49,76\\
                    48,18
                \end{matrix}\right)
\end{equation}

\begin{equation}
    \rho = -0,924
\end{equation}
Die Standardabweichung hat sich verdreifacht und die Korrelation ist nun nahe $-1$.
Das bedeutet, dass wir nun eine Abhängigkeit der zwei Komponenten feststellen, die eigentlich nicht existiert und nur durch unsere Messungenauigkeit hervorgerufen wird.

\subsection*{f)}
Bei $\epsilon = 0.5$ wird $\mathbf{A}$ singulär und die Entfaltung schlägt fehl.