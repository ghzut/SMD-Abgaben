\section{Aufgabe 16}
\label{sec:Aufgabe3}
\subsection*{a)}
Aus der bedingten Wahrscheinlichkeit folgt:

\begin{align*}
p(F|W) &= \frac{P(F\cap\ W)}{P(W)}\\
 &= \frac{\frac{P(F\,\cap\, W)}{P(F)}\cdot P(F)}{P(W)}
 &= \frac{P(W|F)\cdot P(F)}{P(W)}
\end{align*}

\subsection*{b)}
Die Wahrscheinlichkeit, dass Fußball gespielt wird, beträgt
\[
p(F) = \frac{36}{56} = \frac{9}{14} \text{.}
\]
Die Wahrscheinlichkeiten, dass Fußball gespielt wird und die gegebenen Wetterbedingungen herrschen, betragen:
\begin{align*}
p_.{Wind = stark, F} &= \frac{3}{9} &= \frac{1}{3}\\
p_.{Feucht = hoch, F} &= \frac{3}{9} &= \frac{1}{3}\\
p_.{Temp = kalt, F} &= \frac{3}{9} &= \frac{1}{3}\\
p_.{Ausb = sonnig, F} &= \frac{2}{9}
\end{align*}
Die Wahrscheinlichkeiten, dass diese Wetterbedingungen eintreten beträgt:
\begin{align*}
p_.{Wind = stark} &= \frac{6}{14}\\
p_.{Feucht = hoch} &= \frac{1}{2}\\
p_.{Temp = kalt} &= \frac{3}{7}\\
p_.{Ausb = sonnig} &= \frac{3}{14}
\end{align*}
Die Normierung $p(W)$ ist das Produkt der Wahrscheinlichkeiten, dass die heutigen Bedingungen eintreten, also 
\[
p(W) = \frac{54}{2744} = \frac{27}{1372}
\]
Damit ergibt sich
\[
p(F|W) = \frac{\frac{2}{243}}{\frac{27}{1372}}\frac{9}{14}= \frac{196}{729}\approx 26,886\si{\percent}
\]

\subsection*{c)}
Da bei der Berechnung der Gesamtwahrscheinlichkeit mit diesem Ansatz alle Wahrscheinlichkeiten für das Eintreten der Wetterbedingungen multipliziert werden, in diesem Datensatz die Wahrscheinlichkeit $p(Themperatur = heiß) = 0$ ist, ist auch die Gwesamtwahrscheinlichkeit $0$.
Ein größerer Datensatz könnte das Problem beheben oder man nähert die Wahrscheinlichkeit in dem man die Temperatur außer Acht lässt.

%♥☻